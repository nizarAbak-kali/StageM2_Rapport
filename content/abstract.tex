\section{Résumé du stage}

Ce stage s'est déroulé dans la jeune startup NEO-ROBTIX, société dont le projet est d'utiliser des drones lors d'inspection d'ouvrages (bâtiments, ponts, rails, routes...), sous la direction du CTO M. Benoit COURTY. 
Au cours de ce stage, j'ai eu l'opportunité de travailler sur une grandes variétés de projets et d'avoir la chance de pouvoir tester sur le terrain le résultat de mes efforts, et ainsi engranger un maximum de connaissances et d'expériences que je compte utiliser plus tard.
\newline
\newline
Ce stage ce découpe en trois grandes partie. Une première concentré sur la navigation du drone ainsi que le développement des outils nécessaire à l'inspection d'ouvrages. Puis dans un second temps, une partie apprentissage où nous devait reconnaître automatiquement les fissures dans une route. Enfin une partie traitement d'image où nous devions répondre à plusieurs problématique tel que: la détection et le suivie de route par le drone, l'implémentation d'algorithme pour de l'image stitching\footnote{L'assemblage de photos est un procédé consistant à combiner plusieurs images numériques se recouvrant, dans le but de produire un panorama ou une image de haute définition.}, finalement la détection d'obstacle .
\newline
\newline
Ma première tâche fut sous la tutelle de Benoit Courty, le responsable technique de NEO-ROBTIX, de trouver, puis d'implémenter une solution ROS\footnote{Robot Operating System: est un ensemble d'outils informatiques open source permettant de développer des logiciels pour la robotique} de navigation d'un drone près du mur .
\\
\\
Dans un second temps, nous avons tenté de répondre à la problématique de reconnaissance automatique de faille dans la route grâce au deep learning\footnote{L'apprentissage profond (en anglais deep learning) est un ensemble de méthodes d'apprentissage automatique tentant de modéliser avec un haut niveau d’abstraction des données grâce à des architectures articulées de différentes transformations non linéaires}. Pour ce faire nous nous sommes appuyé sur le papier d'une étudiante suédoise qui a utilisé les technologies Caffe\footnote{Caffe est un framework de deep learning développé par le Berkeley Vision and Learning Center} et DIGITS\footnote{Digits est un système développé par Nvidia pour gérer les réseaux de deep learning depuis une interface web}
\\
\\
Dans la dernière partie de mon stage, j'ai eu pour mission de développer une solution de détection de route puis de suivie de celle-ci en temps réel.

